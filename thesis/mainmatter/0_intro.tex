\chapter*{Introduction}

The ambition to comprehend and model the world is inherent to the human condition. In this pursuit, mathematical language has proven to be an invaluable tool, serving as the basis for any physical theory aiming to predict the outcome of a particular process. These physical theories have been refined, expanded, and perfected over the years. Today, there is a suggested theory seeking to predict and understand nearly every physical process.

However, at the end of the nineteenth century, some scientists believed that all known physical theories were sufficient to describe nature. As Lord Kelvin famously stated:
\begin{center}
    \begin{tabular}{p{12cm}}
        \emph{``There is nothing new to be discovered in physics now. All that remains is more and more precise measurement.''}\\
        \hfill{}-- Lord Kelvin (1901)\\
    \end{tabular}
\end{center}
As we know today, he could not have been more wrong, as the last century has been the most scientifically successful in all history. Notable examples of these successful theories are Quantum Mechanics (QM) and the Standard Model of Particle Physics (SM), the latter aiming to explain the behaviour of the most fundamental particles that are thought to constitute all matter.

Quantum Mechanics, born in the early decades of the last century, aimed to explain the wave-particle duality for light, among other observed and unexplained phenomena, like the photoelectric effect. Scientists like Max Planck, Albert Einstein, Erwin Schrödinger and Paul Dirac, along with many other brilliant minds, developed a theory capable of explaining the intricacies of the interaction between particles of light, photons, and matter particles. Quantum Mechanics was subsequently followed by Quantum Field Theory (QFT) when the idea of quantizing the electromagnetic field emerged in the late 1920s. QFT combined three major themes of modern physics: quantum mechanics, the field concept, and the principle of relativity. This set of theories gave rise to quantum electrodynamics (QED) in the 1950s, and its success and predictive power led to efforts to apply the same basic concepts to the other forces of nature. By the late 1970s, these efforts successfully crystallized into the modern Standard Model of Particle Physics, explaining both the strong and electroweak nuclear forces with a single theory. For further details on QFT, QED and the Standard Model, refer to Refs. \cite{Perkins:1982xb, Peskin:1995ev, Schwartz:2014sze}.

The Standard Model assumed certain symmetries in nature, under which the laws of physics remain unchanged. Formally, these symmetry transformations form a group, called gauge symmetry group. The underlying gauge symmetry group governing the SM not only predicted three massive vector bosons --- the W$^\pm$ and the Z$^0$, proposed in 1968 by Sheldon Glashow, Steven Weinberg, Abdus Salam, and other scientists \cite{Glashow:1961tr, Salam:1964ry, Weinberg:1967tq} ---, but also a massive scalar boson, named after Peter Higgs who, along with Robert Brout, François Englert, Gerald Guralnik, C. R. Hagen, and Tom Kibble, proposed it to explain how all massive particles acquire mass \cite{Higgs:1964pj, Englert:1964et, Guralnik:1964eu}.

Only fifteen years after the prediction of the massive gauge vector bosons, in 1983, the W$^\pm$ and Z$^0$ bosons were discovered at CERN by the UA1 and UA2 collaborations \cite{UA1:1983crd, UA2:1983tsx}, establishing the SM as a successful theory combining the weak and electromagnetic interactions. However, the Standard Model was incomplete without the Higgs boson, yet there was no sign of its existence.

Thirty more years of work and technology development were necessary until, on July 4\textsuperscript{th}, 2012, the final piece of the puzzle fell into place. On that day, the CMS and ATLAS collaborations jointly announced the discovery of the Higgs boson at CERN's Large Hadron Collider (LHC) \cite{CMS:2012qbp, ATLAS:2012yve}. Although a reassuring finding, this did not consolidate the SM as the ultimate, immutable theory but marked the beginning of understanding the properties of this newly discovered particle and an opportunity to test the SM to its limits.

One of the predictions of the Standard Model is that the Higgs boson interacts with massive particles with a strength related to the particle's mass. This has been successfully tested with remarkable accuracy for heavy particles: the gauge vector bosons W$^\pm$ and Z$^0$, the heavy members of the third family of fermions (top and bottom quarks, and the tau lepton, but not the neutrino), and even the second-generation muon lepton \cite{CMS:2022dwd}. However, measuring this coupling becomes increasingly challenging as the particle becomes lighter, as it is less likely to interact with the Higgs boson. Testing the SM hypothesis for less massive particles could reveal discrepancies between the theory and experiments, hinting at new physics beyond the SM (BSM), such as the existence of new particles and processes not previously considered.

Currently, the couplings of light quarks --- including up, down, charm and strange --- to the Higgs boson are not well-constrained due to limited data on the Higgs boson's total decay width, which is the rate at which the Higgs boson decays. The presence of a large multi-jet background at the LHC makes it challenging to study these couplings using inclusive Higgs decays into quark-antiquark pairs. To address this, rare exclusive decays of the Higgs boson into a light meson and a photon have been suggested. These exclusive decays serve as a valuable tool for investigating both flavor-conserving and flavor-violating Higgs boson couplings to light quarks. The ATLAS and CMS collaborations have made initial attempts to set upper limits on these hadronic two-body Higgs boson decays, including processes like H$\decaysto \text{J}/\psi + \gamma$ \cite{ATLAS:2022rej, CMS:2018gcm}, H$\decaysto \rho,\phi,\omega,\text{K}^{*0} + \gamma$ \cite{ATLAS:2017gko, ATLAS:2023alf}, and H$\decaysto \text{J}/\psi,\rho,\phi + \text{Z}^0$ \cite{CMS:2022fsq, CMS:2020ggo}.

The analysis presented in this Master Thesis ultimately aims to constrain the Higgs coupling constants by studying four exotic decays of the Higgs boson into a photon and a light vector meson, where the vector meson decays into a pair of charged scalar mesons along with neutral particles, specifically either pions or photons. It uses data from proton-proton collisions corresponding to an integrated luminosity of 39.54 fb$^{-1}$ at $\sqrt{s}=$13 TeV, collected by the CMS detector at the LHC in 2018 during Run 2. The final states of interest consist of an isolated and energetic photon, a charged meson pair, and photons compatible with a third (and sometimes fourth) neutral meson. The mesons considered here are a $\phi$, a $\omega$ and a $\text{D}^{*0}$, each further decaying into two charged particles and a third (and fourth) neutral one:

\begin{table}[!ht]
    \centering
    \begin{tabular}{ll}
        $\text{H}\decaysto \phi\gamma$ ,& $\phi\decaysto \pi^+\pi^-\pi^0$ \\
        $\text{H}\decaysto \omega\gamma$ ,& $\omega\decaysto \pi^+\pi^-\pi^0$\\
        $\text{H}\decaysto \text{D}^{*0}\gamma$ ,& $\text{D}^{*0}\decaysto \text{D}^{0}\pi^{0}/\gamma,\ \text{D}^{0}\decaysto \text{K}^{-}\pi^{+}$\\
        $\text{H}\decaysto \text{D}^{*0}\gamma$ ,& $\text{D}^{*0}\decaysto \text{D}^{0}\pi^{0}/\gamma,\ \text{D}^{0}\decaysto \text{K}^{-}\pi^{+}\pi^{0}$
    \end{tabular}
\end{table}
\newpage
This study aims to place an upper limit on the branching ratio of the four different Higgs boson rare decays under study. The large background at the LHC and the infrequent nature of the decays prevent an actual measurement of the decay rates, allowing only an upper bound to be computed. The expected upper  limits are determined using simulated samples, not real data. This is a common practice in high-energy physics, where data can only be \textit{unblinded} for actual measurements after completing the full analysis and ensuring the consistency of techniques.

Simpler decays of the same nature (H$\decaysto M + \gamma$) are currently being studied by the CMS collaboration. These decay channels are H$\decaysto \rho^0/\phi/\text{K}^{*0} + \gamma$, but where the meson further decays only into a pair of charged scalar mesons. This analysis follows a similar approach, with a focus on reconstructing the additional neutral particles.

The thesis is structured as follows. The first chapter briefly presents the Standard Model of Particle Physics, with a special emphasis on the Higgs boson, its properties, and frequent production and decay modes. It also motivates the analysis, explaining in detail the decay channels under study and exploring how a significant discrepancy between the measurements of these decay modes and the SM predictions might lead to new physics beyond the SM.

Chapter \ref{chap:CMS_LHC} introduces CERN and its Large Hadron Collider, as well as the Compact Muon Solenoid (CMS) experiment. It covers CERN's most significant breakthroughs, with a particular emphasis on the discovery of the Higgs boson at the LHC in 2012 by the CMS and ATLAS collaborations.

Chapter \ref{chap:analysis} is entirely dedicated to the analysis. It begins with a general overview, followed by an explanation of the samples, triggers, and object definitions. A section discusses the meson reconstruction techniques, which are crucial to the analysis. It also covers the criteria used in event selection and provides a comparison between the simulated samples and real data of the most important kinematic variables, before explaining how the signal and background have been modelled. Finally, the expected upper limits of the branching ratio for each channel are presented. The chapter concludes by addressing the subsequent steps required before data unblinding, as well as suggesting ideas to improve the results.

The thesis ends with a summary and conclusions of the conducted study, along with an appendix containing supplemental technical analysis details.
